\documentclass[11pt,a4paper]{article}

\usepackage[T1]{fontenc}
\usepackage{geometry}
\usepackage{enumitem}
\usepackage{titlesec}
\usepackage{xeCJK}
\usepackage{fontspec}
\usepackage{hyperref}
\usepackage{setspace}
\usepackage{multicol}

% Page setup
\geometry{
    left=1.5cm,
    right=1.5cm,
    top=1.5cm,
    bottom=1.5cm
}
\pagestyle{empty}
% Font setup
\setmainfont{Times New Roman}
\setCJKmainfont[
    Path = ./,
    UprightFont = SourceHanSerifCN-Regular.otf,
    BoldFont = SourceHanSerifCN-Bold.otf
]{SourceHanSerifCN}

% Minimal spacing
\setstretch{1.0} 
\setlength{\parindent}{0pt}
\setlength{\parskip}{0.1em}  % Very small paragraph spacing
% Compact section formatting
\titlespacing*{\section}{0pt}{0.4em}{0.2em}
% Section formatting
\titleformat{\section}{\large\bfseries}{}{0em}{}[\titlerule]
\titleformat{\subsection}{\normalsize\bfseries}{}{0em}{}
\titleformat{\subsubsection}{\normalsize\bfseries}{}{0em}{}

\setlist[itemize]{leftmargin=1em, itemsep=0pt, topsep=0pt}

% Custom spacing
\setlength{\parindent}{0pt}

\begin{document}

% Header
\begin{center}
    {\LARGE \textbf{路海阔 Ezra}}\\[0.5em]
    具备统计建模与系统优化双重背景,专注 AI 算法实现与大模型工程应用。
    \\
    熟悉非线性模型、Agent 框架、文档解析等场景,擅长用 C++/Python 构建高性能系统并推动实际落地。
    \vspace{0.5em}
    \rule{\textwidth}{0.4pt}
\end{center}
\begin{multicols}{2}

\section{教育经历}

\subsection{Boston University}
\textbf{Master of Arts in Statistics} \hfill Boston, MA\\
College of Math and Statistics \hfill GPA: 3.58/4.0\\
\textit{2018.09 -- 2020.01}
\begin{itemize}[leftmargin=1em, itemsep=0pt]
    \item 核心课程: Statistical Learning, General Linear Regression, Bayesian Statistics, Computational Statistics, Stochastic Process
\end{itemize}
\subsection{北京理工大学}
\textbf{统计学学士} \hfill 北京\\
数学与统计学院 \hfill GPA 3.3/4.0\\
\textit{2014.09 - 2018.06}
\begin{itemize}[leftmargin=1em, itemsep=0pt]
    \item 核心课程: 数学分析, 高等代数, 实分析, 概率论, 数理统计, C 语言等
\end{itemize}

\section{工作经历}

\subsection{博佳医药科技有限公司}
\textbf{中级算法研究员} | 上海 | 2023.08 – 至今

\textbf{非线性混合效应模型算法开发}:
\begin{itemize}[leftmargin=1em, itemsep=0pt]
    \item 复现前沿论文算法(论文 → Python Demo → C++),结果与行业金标准软件一致
    \item 设计测试用例,进行数据模拟与采集,验证算法准确性
    \item 优化计算性能,减少重复运算,提升运行效率(如通过 BLAS/LAPACK 替换 Eigen,加速 70\%+)
\end{itemize}

\subsection{浙江来未来科技有限公司}
\textbf{数据科学家|数据智能部} | 2021.07 – 2023.08
\begin{itemize}[leftmargin=1em, itemsep=0pt]
    \item \textbf{医疗 AI 项目研发}:主导医疗科研合作项目的 AI 算法设计与实现,涵盖数据采集、模型研发、算法部署等全流程,服务医院临床科研与管理需求。
    \item \textbf{数据搜集及清理流水线搭建}:根据医院临床科室需求, 综合整理不同平台患者数据, 根据多样数据格式搭建对应结构化数据流水线。
    \item \textbf{多模态数据处理}:基于 ResNet,卷积网络等,处理时间序列(如心率波形)、医学图像等多模态医疗数据,完成特征提取与融合。
\end{itemize}

\section{项目经历}

\subsection{FOCE(First Order Condition Estimation) 与 SAEM 算法开发}
\begin{itemize}[leftmargin=1em, itemsep=0pt]
    \item 主导从零实现 FOCEI 与 SAEM 两种主流非线性混合效应模型(NLME)参数估计算法,涵盖 Hessian 近似、梯度计算与优化器集成。
    \item 参考前沿文献对梯度求解流程进行数值与解析混合优化,显著减少迭代耗时,整体加速约 30\%。
    \item 构建了可复用的矩阵与三维数组结构,采用内存池管理,减少 Eigen 内部内存分配瓶颈,在数万重复模拟任务中提升内存利用率与运行效率。
    \item 在多种模型结构与真实数据模拟下,系统性对比输出与 NONMEM、Monolix 等商业软件的一致性,验证算法精度与稳定性。
    \item 完善了模型比较流程、VPC 预测检验图输出与 Bootstrap 置信区间估计,构建完整的建模与评估工作流。
\end{itemize}

\subsection{基于大模型的 PDF 文档结构化信息抽取与自动标注系统}
\begin{itemize}[leftmargin=1em, itemsep=0pt]
    \item \textbf{项目背景}:公司内部大量医疗数据以 PDF 格式存储,涉及临床研究方案、统计报告等,且因数据涉密,需使用私有化部署的大语言模型。统计部门频繁需要从 PDF 文档中提取结构化信息,自动化需求强烈。
    \item \textbf{系统设计与搭建}:使用 Dolphin 及 Deepseek R1 构建从 PDF 到结构化数据的端到端流水线,利用 Dolphin 工具完成版面解析与格式标准化,输出结构化 JSON。
    \item \textbf{大模型标注接口}:将结构化 JSON 输入私有部署的大模型,结合 prompt 编排与 Agent 架构实现实体识别、字段提取与标签生成,自动完成数据标注。
    \item \textbf{实际成效}:系统已在多个临床统计项目中落地,显著减少人工抽取工作量,提高数据提取速度与准确率。
\end{itemize}

\section{个人总结}
\begin{itemize}[leftmargin=1em, itemsep=0pt]
    \item \textbf{编程语言}:Python, C++, R, Bash, SQL
    \item \textbf{常用框架}:PyTorch, NumPy, Pandas, Polars, Scikit-learn, Eigen, SymPy, OpenBLAS/LAPACK
    \item \textbf{建模方法}:FOCEI, SAEM, Bootstrap, Nonlinear Mixed Effects Models, Bayesian Inference
    \item \textbf{发表}: PAGE 31 (2023) Abstr 10676 [\url{www.page-meeting.org/?abstract=10676}]
    \item \textbf{语言能力}:中文(母语),英文(TOEFL 110)
\end{itemize}
\end{multicols}
\end{document}
