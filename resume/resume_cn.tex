\documentclass[11pt,a4paper]{article}

\usepackage[T1]{fontenc}
\usepackage[utf8]{inputenc}
\usepackage{geometry}
\usepackage{enumitem}
\usepackage{titlesec}
\usepackage{CJKutf8}
\usepackage{hyperref}
\usepackage{setspace}
\usepackage{multicol}

% Page setup
\geometry{
    left=1.5cm,
    right=1.5cm,
    top=1.5cm,
    bottom=1.5cm
}
\pagestyle{empty}

% Minimal spacing
\setstretch{1.0}
\setlength{\parindent}{0pt}
\setlength{\parskip}{0.1em}  % Very small paragraph spacing
% Compact section formatting
\titlespacing*{\section}{0pt}{0.4em}{0.2em}
% Section formatting
\titleformat{\section}{\large\bfseries}{}{0em}{}[\titlerule]
\titleformat{\subsection}{\normalsize\bfseries}{}{0em}{}
\titleformat{\subsubsection}{\normalsize\bfseries}{}{0em}{}

\setlist[itemize]{leftmargin=1em, itemsep=0pt, topsep=0pt}

\begin{document}
\begin{CJK*}{UTF8}{gbsn}

% Header
\begin{center}
    {\LARGE \textbf{路海阔 Ezra}}\\[0.5em]
    具备统计建模与系统优化双重背景,专注 AI 算法实现与大模型工程应用。\\
    熟悉非线性模型、Agent 框架、文档解析等场景,擅长用 C++/Python 构建高性能系统并推动实际落地。\\[0.5em]
    \rule{\textwidth}{0.4pt}
\end{center}

\begin{multicols}{2}

\section{教育经历}

\subsection{Boston University}
\textbf{Master of Arts in Statistics} \hfill Boston, MA\\
College of Math and Statistics \hfill GPA: 3.58/4.0\\
\textit{2018.09 -- 2020.01}
\begin{itemize}[leftmargin=1em, itemsep=0pt]
    \item 核心课程: Statistical Learning, General Linear Regression, Bayesian Statistics, Computational Statistics, Stochastic Process
\end{itemize}

\subsection{北京理工大学}
\textbf{统计学学士} \hfill 北京\\
数学与统计学院 \hfill GPA 3.3/4.0\\
\textit{2014.09 -- 2018.06}
\begin{itemize}[leftmargin=1em, itemsep=0pt]
    \item 核心课程: 数学分析, 高等代数, 实分析, 概率论, 数理统计, C 语言等
\end{itemize}

\section{工作经历}

\subsection{博佳医药科技有限公司}
\textbf{中级算法研究员} \hfill 上海 \hfill 2023.08 -- 至今
\begin{itemize}[leftmargin=1em, itemsep=0pt]
    \item 复现前沿 NLME 算法(论文 $\rightarrow$ Python Demo $\rightarrow$ C++),结果与行业金标准软件一致。
    \item 设计测试用例并搭建模拟流程,验证多种剂量方案下的算法准确性。
    \item 优化计算性能,减少重复运算,通过 BLAS/LAPACK 替换 Eigen 关键模块,加速 70\% 以上。
\end{itemize}

\subsection{浙江来未来科技有限公司}
\textbf{数据科学家|数据智能部} \hfill 2021.07 -- 2023.08
\begin{itemize}[leftmargin=1em, itemsep=0pt]
    \item 主导医疗 AI 合作项目的算法设计与实现,覆盖数据采集、模型研发与部署,服务医院科研需求。
    \item 针对多源数据搭建自动化采集与清洗流水线,满足临床科室的结构化数据需求。
    \item 处理心率波形、医学图像等多模态数据,基于 ResNet/CNN 完成特征提取与融合。
\end{itemize}

\section{项目经历}

\subsection{FOCE 与 SAEM 算法开发}
\begin{itemize}[leftmargin=1em, itemsep=0pt]
    \item 独立实现 FOCEI 与 SAEM 两类 NLME 参数估计算法,涵盖 Hessian 近似、梯度计算与优化器集成。
    \item 结合数值与解析混合策略提升梯度求解效率,迭代耗时下降约 30\%。
    \item 构建可复用矩阵与张量结构,采用内存池管理减少 Eigen 内存分配瓶颈,提升模拟效率。
    \item 对比 NONMEM、Monolix 等商业软件输出,验证算法精度与稳定性。
\end{itemize}

\subsection{大模型驱动的 PDF 结构化抽取与自动标注}
\begin{itemize}[leftmargin=1em, itemsep=0pt]
    \item 搭建从 PDF 到标准化 JSON 的全流程,将版面解析与格式归一化前置于 LLM 标注。
    \item 使用私有部署的 Dolphin 与 DeepSeek-R1,结合 Prompt 编排与 Agent 逻辑实现实体提取与标签生成。
    \item 系统在多个临床统计项目落地,显著降低人工标注成本并提升准确率。
\end{itemize}

\section{个人总结}
\begin{itemize}[leftmargin=1em, itemsep=0pt]
    \item \textbf{编程语言}: Python, C++, R, Bash, SQL
    \item \textbf{常用框架}: PyTorch, NumPy, Pandas, Polars, Scikit-learn, Eigen, SymPy, OpenBLAS/LAPACK
    \item \textbf{建模方法}: FOCEI, SAEM, Bootstrap, NLME, Bayesian Inference
    \item \textbf{发表}: PAGE 31 (2023) Abstract 10676 \href{https://www.page-meeting.org/?abstract=10676}{www.page-meeting.org/?abstract=10676}
    \item \textbf{语言能力}: 中文(母语),英文(TOEFL 110)
\end{itemize}

\end{multicols}

\end{CJK*}
\end{document}
